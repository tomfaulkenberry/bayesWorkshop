% Created 2019-04-06 Sat 06:23
% Intended LaTeX compiler: pdflatex
\documentclass[11pt]{article}
\usepackage[utf8]{inputenc}
\usepackage[T1]{fontenc}
\usepackage{graphicx}
\usepackage{grffile}
\usepackage{longtable}
\usepackage{wrapfig}
\usepackage{rotating}
\usepackage[normalem]{ulem}
\usepackage{amsmath}
\usepackage{textcomp}
\usepackage{amssymb}
\usepackage{capt-of}
\usepackage{hyperref}
\usepackage[left=1in,right=1in,top=1in,bottom=1in]{geometry}
\usepackage{fancyhdr}
\lhead{Example Bayesian Writeups}
\rhead{Tom Faulkenberry (faulkenberry@tarleton.edu)}
\pagestyle{fancy}
\parskip = 0.1in
\date{}
\title{}
\hypersetup{
 pdfauthor={},
 pdftitle={},
 pdfkeywords={},
 pdfsubject={},
 pdfcreator={Emacs 26.1 (Org mode 9.1.9)}, 
 pdflang={English}}
\begin{document}



\section*{1. Anagram task}
\label{sec:orga6576f0}
\subsection*{Basic}
\label{sec:orgd26c79f}
"We defined two models to describe our data: \(\mathcal{H}_1\) states that mean solution time for participants in the organized list condition will be less than the mean solution time for participants in the unrelated list condition, whereas \(\mathcal{H}_0\) states that mean solution time will not differ between conditions. We then computed a Bayesian independent samples \(t\)-test (Rouder et al., 2009) to quantify the evidence for \(\mathcal{H}_1\) over \(\mathcal{H}_0\).  We found a Bayes factor of \(B_{10}=7.11\), indicating that the observed data are approximately 7 times more likely under \(\mathcal{H}_1\) than \(\mathcal{H}_0\)."

\subsection*{Advanced}
\label{sec:org31ad35a}
"We defined two models to describe our data: \(\mathcal{H}_1:\delta < 0\) states that mean solution time for participants in the organized list condition will be less than the mean solution time for participants in the unrelated list condition, whereas \(\mathcal{H}_0:\delta=0\) states that mean solution time will not differ between conditions. We then computed a Bayesian independent samples \(t\)-test (Rouder et al., 2009) to quantify the evidence for \(\mathcal{H}_1\) over \(\mathcal{H}_0\). This test requires the user to specify a prior distribution for effect size \(\delta\), which we initially took at the default Cauchy prior with scale \(r=0.707\). Using this prior, we found a Bayes factor of \(B_{10}=7.11\), indicating that the observed data are approximately 7 times more likely under \(\mathcal{H}_1\) than \(\mathcal{H}_0\).  Additionally, we performed a robustness check by varying the prior scale factor \(r\), each reflecting a different \emph{a priori} expectation of the effect of our manipulation. Generally, \(B_{10}\) decreases as the scale factor \(r\) increases, but even using a very wide prior with \(r=1.41\), the inference remains the same, as the data are approximately 4.5 times more likely under \(\mathcal{H}_1\) than under \(\mathcal{H}_0\). "

\section*{2. Kitchen rolls experiment}
\label{sec:org1e63c47}
\subsection*{Basic}
\label{sec:orgd95ffff}
"We defined two models to describe our data: \(\mathcal{H}_1\) states that mean NEO score for participants in the clockwise condition will be greater than the mean NEO score for participants in the counterclockwise condition, whereas \(\mathcal{H}_0\) states that mean NEO score will not differ between conditions. We then computed a Bayesian independent samples \(t\)-test (Rouder et al., 2009) to quantify the evidence for \(\mathcal{H}_0\) over \(\mathcal{H}_1\).  We found a Bayes factor of \(B_{01}=7.80\), indicating that the observed data are approximately 7 times more likely under \(\mathcal{H}_0\) than \(\mathcal{H}_1\)."

\subsection*{Advanced}
\label{sec:org63bbefb}
"We defined two models to describe our data: \(\mathcal{H}_1:\delta > 0\) states that mean NEO score for participants in the clockwise condition will be greater than the mean NEO score for participants in the counterclockwise condition, whereas \(\mathcal{H}_0:\delta=0\) states that mean NEO score will not differ between conditions. We then computed a Bayesian independent samples \(t\)-test (Rouder et al., 2009) to quantify the evidence for \(\mathcal{H}_0\) over \(\mathcal{H}_1\). This test requires the user to specify a prior distribution for effect size \(\delta\), which we initially took at the default Cauchy prior with scale \(r=0.707\). Using this prior, we found a Bayes factor of \(B_{01}=7.80\), indicating that the observed data are approximately 7 times more likely under \(\mathcal{H}_0\) than \(\mathcal{H}_1\).  Additionally, we performed a robustness check by varying the prior scale factor \(r\), each reflecting a different \emph{a priori} expectation of the effect of our manipulation. \(B_{01}\) increases as the scale factor \(r\) increases (\(BF_{01}=10.8\) with scale \(r=1\) and \(BF_{01}=15.1\) with scale \(r=1.41\)). Thus, our data is strongly evidential of \(\mathcal{H}_0\) over \(\mathcal{H}_1\)."


\section*{3. Recall task}
\label{sec:org7936ff9}

" We defined five competing models for our 2x2 factorial design:
\begin{itemize}
\item \(\mathcal{M}_0\): null model
\item \(\mathcal{M}_1\): encodingCue only
\item \(\mathcal{M}_2\): retrievalCue only
\item \(\mathcal{M}_3\): encodingCue + retrievalCue
\item \(\mathcal{M}_4\): encodingCue + retrievalCue + encodingCue*retrievalCue
\end{itemize}

These models were compared via a Bayesian analysis of variance (Rouder et al., 2012).  Equal prior probabilities were assigned to the five competing models.  The additive model received the largest posterior probability, \(p(\mathcal{M}_4\mid \text{data}) = 0.621\).  This was larger than the posterior probability of the interactive model, which received a posterior probability of \(p(\mathcal{M}_4\mid \text{data})=0.156\), as well as the posterior probability of the retrieval cue only model, \(p(\mathcal{M}_2\mid \text{data})=0.179\). In terms of model odds, the additive model received the most support from the data, which updated the prior odds for the model by a factor of 6.56.  In comparison, the prior odds for all other models were actually decreased by the data.  Finally, the Bayes factor directly comparing \(\mathcal{M}_3\) to \(\mathcal{M}_4\) was equal to 3.99, indicating that the observed data is approximately 4 times more likely under the additive model than under the interactive model.''


\section*{4. Mental arithmetic task}
\label{sec:orga9838f0}

" We defined five competing models for our 2x2 factorial design:
\begin{itemize}
\item \(\mathcal{M}_0\): null model
\item \(\mathcal{M}_1\): format only
\item \(\mathcal{M}_2\): problem size only
\item \(\mathcal{M}_3\): format + problem size
\item \(\mathcal{M}_4\): format + problem size + format*problem size
\end{itemize}

These models were compared via a Bayesian analysis of variance (Rouder et al., 2012).  Equal prior probabilities were assigned to the five competing models.  The interactive model received the largest posterior probability, \(p(\mathcal{M}_4\mid \text{data}) = 0.998\).  This was much larger than the posterior probability of the additive model, which received a posterior probability of \(p(\mathcal{M}_3\mid \text{data})=0.002\). In terms of model odds, the interactive model received the most support from the data, which updated the prior odds for the model by a factor of 2,385.  In comparison, the prior odds for the additive model were actually decreased by a factor of 1/0.007 = 143.  Finally, the Bayes factor directly comparing \(\mathcal{M}_4\) to \(\mathcal{M}_3\) was equal to 596.2, indicating that the observed data is approximately 600 times more likely under the interactive model than under the additive model.''

\section*{References:}
\label{sec:org9984a7b}

\begin{enumerate}
\item Rouder, J. N., Speckman, P. L., Sun, D., Morey, R. D., \& Iverson, G. (2009). Bayesian t tests for accepting and rejecting the null hypothesis. \emph{Psychonomic bulletin \& review, 16}, 225-237.
\item Rouder, J. N., Morey, R. D., Speckman, P. L., \& Province, J. M. (2012). Default Bayes factors for ANOVA designs. \emph{Journal of Mathematical Psychology, 56}, 356-374.
\end{enumerate}
\end{document}